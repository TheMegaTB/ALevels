\documentclass[a4paper]{article}

\usepackage{hyperref}
\usepackage{amsmath}
\usepackage{empheq}
\usepackage{mathtools}
\usepackage[german]{babel}
\usepackage[utf8]{inputenc}
\usepackage{amsfonts}
\usepackage{graphicx}
\usepackage{siunitx}
\usepackage{tabularx}
\usepackage{wrapfig}
\usepackage{float}
\usepackage[margin=2cm]{geometry}


%opening
\title{}
\author{}
\title{Abiturvorbereitung Mathe}
\author{Til Blechschmidt\\ Schüler, Otto-Hahn-Gymnasium Geesthacht, Germany}


\begin{document}

	\pagenumbering{gobble}
	\maketitle
	\newpage
	\pagenumbering{arabic}
	
	\tableofcontents
	\listoffigures
	\listoftables
	\newpage

	\section{Analysis}
		\subsection{Kurvendiskussion}
			\subsubsection{Nullstellen}
				Eine Nullstelle beschreibt den Ort, an dem ein Graph die X-Achse schneidet, also seine Y-Komponente null entspricht.
				Zur Berechnung der Nullstellen einer Funktion muss $f(x) = 0$ gelten. Löst man diese nun nach x auf erhält man die X-Komponente des Nullpunkts.
				
				\begin{equation} \label{null:function}
				f(x) = 5x^2 + 4x
				\end{equation}
				
				Setzt man nun die Funktion \ref{null:function} gleich null so erhält man:
				
				\begin{equation}
					\begin{split}
						f(x) &= 0\\
						5x^2 + 4x &= 0\\
						x(5x + 4) &= 0
					\end{split}
				\end{equation}
				
				Nun kann man den folgenden Satz anwenden: Ein Produkt ist null, wenn einer der Faktoren null ist.\\
				
				\begin{align}
					x_1 &= 0\\
					5x_2 + 4 &= 0 \label{null:second_factor}
				\end{align}
				
				Löst man nun Gleichung \ref{null:second_factor} nach $x$ auf, so erhält man schlussendlich die beiden X-Koordinaten der Nullpunkte
				
				\begin{equation}
					\begin{split}
						5x_2 + 4 &= 0\\
						5x_2 &= -4\\
						x_2 &= -\frac{4}{5}
					\end{split}
				\end{equation}
				
				\noindent
				Nun kann man diese X-Komponente mit der Y-Komponente, welche im Falle der Nullpunkte immer null entspricht, zu einem Punkt zusammenschließen.
				
				\begin{equation}
					N_1\left(0\middle|0\right); 
					N_2\left(0\middle|-\frac{4}{5}\right)
				\end{equation}
				
			\subsubsection{Ableitungen}
				\begin{equation}
					\begin{split}
						f(x) &= ax^b + c\\
						g(x) &= 10x^2 + 3
					\end{split}
				\end{equation}
				
				Die erste Ableitung eines Graphen gibt die Steigung dessen an.
				\begin{subequations}
					\begin{equation}
						f(x)' = a * bx^{b-1}
					\end{equation}
					\begin{equation}
						\begin{split}
							g(x)' &= 10*2x^1\\
							g(x)' &= 20x\label{abl:first}
						\end{split}
					\end{equation}
				\end{subequations}
				
				Die zweite Ableitung bildet sich, indem man die erste Ableitung ableitet, also die Steigung der ersten Ableitung angibt.
				\begin{equation}
					\begin{split}
						g(x)'' &= 20\label{abl:second}
					\end{split}
				\end{equation}
				
				Folglich bildet sich die dritte Ableitung, indem man die zweite ableitet.
				\begin{equation}
					\begin{split}
						g(x)''' &= 0
					\end{split}
				\end{equation}
				
			\subsubsection{Extremstellen}
				Extremstellen beschreiben Orte, an denen der Graph einer Funktion $f(x)$ seine maximale Auslenkung erreicht. Zur Berechnung der Extremstelle einer Funktion muss $f(x)' = 0$, sowie $f(x)'' \neq 0$ gelten. Anschaulich bedeutet es, dass die Steigung des Graphen null entspricht und seine Ausrichtung ist nicht parallel zur X-Achse.
				Folglich setzt man nun die erste Ableitung (\ref{abl:first}) mit null gleich.
				
				\begin{equation}
					\begin{split}
						g(x)' &= 0\\
						20x &= 0\\
						x &= \frac{0}{20}\\
						x &= 0\label{extreme:first}
					\end{split}
				\end{equation}
				
				Nun muss man die zweite, hinreichende Bedingung überprüfen, indem man die zweite Ableitung (\ref{abl:second}) an der zuvor berechneten Stelle berechnet.
				
				\begin{equation}
					\begin{split}
						g(0)'' \neq 0\\
						20 \neq 0
					\end{split}
				\end{equation}
				
				Um den dazugehörige Extrempunkt zu berechnen kann man nun die zuvor in Gleichung \ref{extreme:first} berechnete X-Komponente in die Funktion $f(x)$ einsetzen.
				
				\begin{equation}
					\begin{split}
						f(0) &= y\\
						10 * 0^2 + 3 &= y\\
						3 &= y
					\end{split}
				\end{equation}
				
				\begin{equation}
					E_1\left(0\middle|3\right)
				\end{equation}
				
			\subsubsection{Wendestellen}
			\subsubsection{Symmetrie}
				\subparagraph{Y-Achse}
				\subparagraph{Punktsymmetrie}
		\subsection{Ganz-rationale Funktionen}
		\subsection{Integralrechnung}
			\subsubsection{Fläche zwischen Graph und X-Achse}
			\subsubsection{Fläche zwischen zwei Funktionsgraphen}
		\subsection{Extremwertaufgaben}

	\section{Stochastik}
		\subsection{Baumdiagramm}
		\subsection{Vierfeldertafel}
		\subsection{Verteilungen}
			\subsubsection{Binomialkoeffizient}
				Ein Binomialkoeffizient ist wie folgt definiert.
				\begin{equation}
					\binom{n}{k} = \frac{n!}{k! (n-k)!}
				\end{equation}
			\subsubsection{Binomialverteilung}
				Die Binomialverteilung gibt die Anzahl der Erfolge in einer Serie von Versuchen an, bei denen sich die Wahrscheinlichkeiten für die einzelnen Ereignisse nicht verändert. Dabei können die Versuche nur zwei mögliche Ergebnisse haben. Solche Versuchsreihen werden als Bernoulli-Kette beschrieben.\\
				Um die Binomialverteilung $B(X = k)$ zu berechnen, benötigt man die Erfolgswahrscheinlichkeit $p$ sowie die Anzahl der Versuche $n$ und die Anzahl der zu erwartenden Erfolge $k$.
				\begin{equation}
					B(n; p; k) = \binom{n}{k} * p^k * (1 - p)^{n-k}
				\end{equation}
			\subsubsection{hypergeometrische Verteilung}
				Die hypergeometrische Verteilung gibt die Anzahl der Erfolge in einer Serie von Versuchen an, bei denen sich die Wahrscheinlichkeiten für die einzelnen Ereignisse im Laufe der Versuchsreihe ändern. Zur Berechnung benötigt man:
				\begin{itemize}
					\item[$N$] Anzahl der Elemente (z.B. Kugeln)
					\item[$K$] Anzahl der positiven Elemente (z.B. weißen Kugeln)
					\item[$n$] Anzahl der Züge
					\item[$k$] Anzahl der gezogenen Erfolge
				\end{itemize}
				
				\noindent
				Mit diesen Variablen ist die hypergeometrische Verteilung wie folgt definiert.
				\begin{equation}
					P\left(X = k\right) = \frac{
						\binom{K}{k}
						\binom{N-K}{n-k}
					}{
						\binom{N}{n}
					}
				\end{equation}
			
		\subsection{Erwartungswert}
			Der Erwartungswert gibt den Wert an, welcher zu erwarten ist, wenn man das gegebene Experiment n-Mal durchführt wird. Er wird mit $E\left(X\right)$ bzw. $\mu$ beschrieben und ist wie folgt definiert.
			\begin{equation}
				E(X) = \mu = a_1 * P(X = a_1) + a_2 * P(X = a_2) + \dotsc + a_m * P(X = a_m)
			\end{equation}
			
			\noindent
			Bei der Binomialverteilung kann man den Erwartungswert wie folgt ausdrücken.
			\begin{equation}
				E(X) = n * p
			\end{equation}
		\subsection{Varianz}
			Die Varianz gibt die zu erwartende quadratische Abweichung vom Erwartungswert $\mu$ an und wird mit $V\left(X\right)$ beziehungsweise $\sigma$ beschrieben. Man kann sie mittels folgender Gleichung berechnen.
			\begin{equation}
				V(X) = \sigma^2 = (a_1 - \mu)^2 * P(X = a_1) + (a_2 - \mu)^2 * P(X = a_2) + \dotsc + (a_m - \mu)^2 * P(X = a_m)
			\end{equation}
			
			\noindent
			Bei der Binomialverteilung kann man die Varianz wie folgt ausdrücken.
			\begin{equation}
				\begin{split}
					V(X) &= n * p * q\\
					V(X) &= n * p * (1-p)
				\end{split}
			\end{equation}
		\subsection{Standardabweichung}
			Die Standardabweichung gibt die Abweichung der Werte vom Erwartungswert an. Sie ist die Quadratwurzel der Varianz.
			\begin{equation}
				\sigma = \sqrt{V(X)}
			\end{equation}
		\subsection{Sigma-Regel}
			Mittels der Sigma-Regeln können die Wahrscheinlichkeiten der Umgebung des Erwartungswertes näherungsweise bestimmt werden. Diese treffen jedoch nur zu, sofern die Laplace-Bedingung erfüllt ist ($\sigma > 3$).
			\begin{subequations}
				\begin{align}
					P(\mu - 1\sigma \leq X \leq \mu + 1\sigma) &\approx 0.683\\
					P(\mu - 2\sigma \leq X \leq \mu + 2\sigma) &\approx 0.955\\
					P(\mu - 3\sigma \leq X \leq \mu + 3\sigma) &\approx 0.997
				\end{align}
			\end{subequations}
			\begin{subequations}
				\begin{align}
				P(\mu - 1.64\sigma \leq X \leq \mu + 1.64\sigma) &\approx 0.90\\
				P(\mu - 1.96\sigma \leq X \leq \mu + 1.96\sigma) &\approx 0.95\\
				P(\mu - 2.58\sigma \leq X \leq \mu + 2.58\sigma) &\approx 0.99w
				\end{align}
			\end{subequations}
			

\end{document}          
